\documentclass[11pt,letterpaper]{article}
\usepackage{fullpage}
\usepackage{datetime}
\usepackage[pdftex]{graphicx}
\usepackage{amsfonts,eucal,amsbsy,amsopn,amsmath}
\usepackage{url}
\usepackage[sort&compress]{natbib}
\usepackage{natbibspacing}
\usepackage{latexsym}
\usepackage{wasysym} 
\usepackage{rotating}
\usepackage{fancyhdr}
\DeclareMathOperator*{\argmax}{argmax}
\DeclareMathOperator*{\argmin}{argmin}
\usepackage{sectsty}
\usepackage[dvipsnames,usenames]{color}
\usepackage{multicol}
\definecolor{orange}{rgb}{1,0.5,0}
\usepackage{multirow}
\usepackage{sidecap}
\usepackage{caption}
\renewcommand{\captionfont}{\small}
\setlength{\oddsidemargin}{-0.04cm}
\setlength{\textwidth}{16.59cm}
\setlength{\topmargin}{-0.04cm}
\setlength{\headheight}{0in}
\setlength{\headsep}{0in}
\setlength{\textheight}{22.94cm}
\allsectionsfont{\normalsize}
\newcommand{\ignore}[1]{}
\newenvironment{enumeratesquish}{\begin{list}{\addtocounter{enumi}{1}\arabic{enumi}.}{\setlength{\itemsep}{-0.25em}\setlength{\leftmargin}{1em}\addtolength{\leftmargin}{\labelsep}}}{\end{list}}
\newenvironment{itemizesquish}{\begin{list}{\setcounter{enumi}{0}\labelitemi}{\setlength{\itemsep}{-0.25em}\setlength{\labelwidth}{0.5em}\setlength{\leftmargin}{\labelwidth}\addtolength{\leftmargin}{\labelsep}}}{\end{list}}

\bibpunct{(}{)}{;}{a}{,}{,}
\newcommand{\nascomment}[1]{\textcolor{blue}{\textbf{[#1 --NAS]}}}


\pagestyle{fancy}
\lhead{}
\chead{}
\rhead{}
\lfoot{}
\cfoot{\thepage~of \pageref{lastpage}}
\rfoot{}
\renewcommand{\headrulewidth}{0pt}
\renewcommand{\footrulewidth}{0pt}


\title{11-712:  NLP Lab Report}
\author{Rajarshi Das}
\date{\today}

\begin{document}
\maketitle
\begin{abstract}
%\nascomment{one paragraph here summarizing what the paper is about}
\noindent This is a report on the development of an open source dependency parser for the language, Bengali. Presently I have reported some basic information about the language.
\end{abstract}

\noindent The goal of this project is to design, implement and evaluate a dependency parser for the language, Bengali (also my native language). This language is characterized by a rich system of inflections, derivation and compound formation \citep{saha2004computer,chakroborty2003uchchotoro} which makes analysis and generation of Bengali, a challenging task \citep{ghosh2009dependency}.

\section{Basic Information about Bengali}
According to \citep{ethnologue}, Bengali is an eastern Indo-Aryan Language and is native to the region of eastern south Asia. It is the official language of Bangladesh and is also spoken in the Indian state of West Bengal and parts of Tripura and Assam.\\

\noindent Bengali follows the SOV order in terms of ordering of subject, object and verb \citep{Dasgupta-2003}. It makes use of postpositions instead of prepositions. Determiners follow the noun while numerals, adjectives and possessors precede the noun. It exhibits no case or number agreement and no grammatical gender phenomena \citep{Dasgupta-2003}. Nouns and pronouns are declined into four cases - nominative, objective, genitive and locative \citep{Bhattacharya}\\

\noindent Bengali is written using the Bengali script. It has 11 vowel graphemes and 39 graphemes representing consonants and other modifiers. The script is written and read horizontally from left to right. Figure~\ref{vowels} and ~\ref{cons} show the vowels (and its  various diacritics) and consonants in the Bengali script (Image source: Internet).
\graphicspath{ {images/} }
\begin{figure}[h]
  \caption{Vowels and vowel diacritics in Bengali script.}
  \centering
  \includegraphics[scale=0.35]{vowels}
  \label{vowels}
\end{figure}
\begin{figure}[h]
  \caption{Consonants in Bengali script.}
  \centering
  \includegraphics[scale=0.35]{consonants}
  \label{cons}
\end{figure} \\

\section{Past work on Bengali dependency parsing}

Some work has been done in building dependency parsers for Bengali. \citep{ghosh2009dependency} have used a statistical CRF based model followed by a rule based post processing technique. \citep{Nivre_parsingindian}, \citep{ambati_09} used a transition based dependency parsing model based on MaltParser \citep{Nivre05maltparser:a}. \citep{De_Dep_ben} uses a hybrid approach where they simplify the complex and compound sentential structures and then recombine the parses of the simpler structure by satisfying the demands of the verb groups. \citep{bidir_parser} use a bidirectional parser with perceptron learning with rich context as features. \citep{kosaraju_10} used Maltparser and explored the effectiveness of local morphosyntactic features chunk features and automatic semantic information. \citep{attardi_10} used a transition based dependency shift reduce parser which used a Multi layer Perceptron classifier. They were all tested on the same dataset as a part of a shared task held at ICON 2009 and 2010. \citep{husain_09, husain_10}. In the 2009 contest, \citep{ambati_09} system performed the best and in 2010, best score of Unlabeled Attachment Accuracy was achieved by \citep{attardi_10} and the best scores for Label Accuracy and Labeled Attachment was achieved by  \citep{kosaraju_10}.

\section{Existing useful resources for the task}
Microsoft Research India has a POS tagged dataset for several Indian languages including Bengali. The bengali dataset has 899 POS tagged sentences. Also I have been able to gain access to the annotated dataset which was used in the shared task held at ICON 2009 and 2010. Although I am aware that I cannot use the annotated dataset, I am hopeful that it will provide important insights for annotation.

\section{Attested phenomena in the language}
As mentioned earlier Bengali, like many Indian Languages is a free word order language. There has been an annotation effort for dependency parsing in Bengali in the past as a part of the shared task held at ICON 2009 and 2010. The data was annotated using the computational Paninian Grammar \citep{Bharati}. The paninian grammatical model treats a sentence as a series of modifier- modified elements starting from a primary modified (the root of the tree - generally the main verb) \citep{Bharati-2009}. Also in \citep{Bharati-2009} and \citep{Begum-2008}, they have catalogued in detail all the annotation rules. I am planning to follow the same rules just to be consistent, so that my annotations can be reused by researchers. Although the Paninan theory was formulated by Panini (a grammarian from Ancient India) 2500 years ago for the language Sanskrit, it is basically a dependency grammar \citep{Kiparsky, Shastri}. The framework is inspired by a inflectionally rich language such as Sanskrit and gives a strong framework for annotating for other Indian Languages. Also, although \citep{Bharati-2009} has been written as a guideline for annotating Hindi treebank, similar rules should apply to Bengali, because of the similarity in the languages.

\section{Annotations of test corpus}
For the test corpora, I have chosen a dataset of transcribed text of a speech corpus \citep{Shruti}. The text are from news papers and story books. The transcription of the text has been done carefully and is in ITRANS format. I have also found a part of the dataset tagged with the corresponding POS tags. This has really been helpful while annotating for dependencies.

Some of the annotation rule followed:

\begin{enumerate}
\item Multi-word name or proper nouns: In this case, I have made the last word of the name as the root of the chunk and the tree is a linear chain with the first word being the leaf. For example, the proper noun, Mr. Ramesh Singh would become Mr $\leftarrow$ Ramesh $\leftarrow$ Singh.
\item In Bengali, sometimes adverbs are repeated in order to stress something. In this case we again form a linear chain as above. This time the root is the first occurrence of the adverb.
\item Relative clauses - TBD
\item Negative particles: In many cases, negative words normally group with the verb to change the sentence. In Bengali, the negative word usually is after the verb. I have annotated this chunk with the verb as the parent and the negative word as the child.
\item In many cases, Bengali has a lot of multi verb expression (verbs occurring together and expressing the same thing). In such cases I have also annotated the dependency as a linear chain with the head as the first occurring verb.
\item More to be documented.
\end{enumerate}

For the strategy of implementation, I am thinking of doing a  mix of semi-supervised and rule based methods.

\section{First Round of evaluation}
I have been able to annotate around 3000 tokens in total. Therefore I trained a supervised parser (Turbo Parser) with 1000 training examples. On test dataset A (1000 tokens), it gave an unlabeled attachment score of 41.6$\%$. This is very low and I think because this is primarily because of the very less number of training examples. I plan to annotate a lot more tokens and also try to develop a rule based parser.
 

\newpage


\bibliographystyle{plainnat}
\bibliography{refs}
\label{lastpage}
\end{document}
